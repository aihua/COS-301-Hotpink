%
%Latex Document made by Team HotPink for SRS, Round 1, Assignment 1, COS 301 2017
%

\documentclass[12pt,a4paper]{report}

\title{Software Requirements Specification \\ NavUP \\ COS 301 \\ Team: Hotpink}
\date{Friday, February 24, 2017}
\author{Gregory Austin 14039712 \\ Stephanie Groutsch 14293324 \\ Timothy Kirker 11152402 \\ Drew Langley 11039753 \\ Xoliswa Ntshingila 13410378 \\ Melvin Zitha 12138747}


\usepackage[latin1]{inputenc}
\usepackage{amsmath}
\usepackage{amsfonts}
\usepackage{amssymb}
%\usepackage{fullpage}

\begin{document}
\maketitle


%insert table of contents and figues etc

	%%%%%%%%%%%%%%%%%%%%%%%%%%%%%%%%%%%%%%%%%%%%%%%%%%%%%%%%%%%%%%%%%%%%%%%%%%%%%%
	%%%%%%%%%%%%%%%%%%%%%%%		INTRODUCTION	     %%%%%%%%%%%%%%%%%%%%%%%%%%%%%%%%%%%%%%%%%%
	%%%%%%%%%%%%%%%%%%%%%%%%%%%%%%%%%%%%%%%%%%%%%%%%%%%%%%%%%%%%%%%%%%%%%%%%%%%%%%
\newpage
 \section*{Introduction}
The class of COS 301 2017 have been assigned the NavUP project, an application that will assist students in their navigation around the University of Pretoria.

	\subsection*{Purpose}
	The purpose of this document is to provide a detailed description of the requirements for the NavUP application. It will outline and explain the complete plan for building the NavUP system. 
	Among these explanations will include the constraints for the system and how the system will interact with external applications. The primary purpose of this document is to be used for client approval such that the system may be implemented.

		
	\subsection*{Scope}
		
		
	\subsection*{Definitions, Acronyms and Abbreviations}
	
	\subsection*{References}
	
	
	\subsection*{Overview}
		

	%%%%%%%%%%%%%%%%%%%%%%%%%%%%%%%%%%%%%%%%%%%%%%%%%%%%%%%%%%%%%%%%%%%%%%%%%%%%%%
	%%%%%%%%%%%%%%%%%%%%%%%		OVERALL DESCRIPTION	     %%%%%%%%%%%%%%%%%%%%%%%%%%%%%%%%%%%%%%%
	%%%%%%%%%%%%%%%%%%%%%%%%%%%%%%%%%%%%%%%%%%%%%%%%%%%%%%%%%%%%%%%%%%%%%%%%%%%%%%
\newpage
\section*{Overall Description}
	\subsection*{Product Perspective}
	\subsection*{Product Functions}
	The NavUP mobile application will support a variety of functions namely navigation, providing information, allowing for users to decide on different routes, achieving goals and personalistaion capabilities.
\\

The navigation functionality will consist of locating the user and then navigating them to their required location. They will be provided with searching capabilities in order to obtain their required location. Users will then be able to save their destination for future use.
\\

Information will be provided to users when they are near points of interest. This information will consist of historical and general information regarding venues or landmarks. Information regarding cultural and sporting events that relate to the user will also be shown when appropriate.
\\

Different routes will be provided based on the users needs. Such routes will include the fastest path from one location to the next. This path will be determined based on the amount of pedestrian traffic. Another route will be provided for special needs users in order to allow them to reach their destination using the easiest path for their disability. 
\\

Weekly goals will be provided to motivate users to attend all their lectures during the week as well as to increase the amount of steps they take.

\subsection*{User Characteristics}
There are three main categories of individuals that will make use of the software namely administrators, general users and information providers.
	\\		

		
	The administrators will ensure that the software works as intended and will be in charge of maintaining and upgrading the application. These users will need to understand how the system works as well as have programming knowledge in order to work on the software.  
	\\

	The general users of the system will consist of students, lecturers, guests and university employees. They will only make use of the mobile application. This means that they will not require any knowledge of how the software is implemented however, they will need to have knowledge regarding how to use a mobile phone. These users will mainly use the application to navigate them from one place in the university to the next.
	\\

	The information providers role will be to inform administrators regarding events taking place at the university as well as to provide information regarding different buildings and landmarks within the university. They will not necessarily make use of the mobile application and will therefore only require knowledge regarding the university itself.
	
	\subsection*{Constraints}
	The following are restrictions related to the application:
		\begin{enumerate}
				\renewcommand{\labelenumi}{{\textbf{\arabic{enumi}.}}}
				\item Security  - The users personal information and current location should not be accessible to the public.
				\item Accuracy - The users location should be found whether the user is indoors or not. The location of the user should also be found in terms of what floor of a building they are on.
				\item Performance - The system should be able to handle a large amount of users making use of the software at the same time.
				\item Reliability - The application should still operate when one or more data access points are no longer available.
				\item Accessibility - The application should be easily accessible to everyone that requires it.
				\item Usability - The applications interface should be easy to use and understand.
				\item Size - The application should not require too much memory in order to operate.
				\end{enumerate}
	\subsection*{Assumptions and Dependencies}

	%%%%%%%%%%%%%%%%%%%%%%%%%%%%%%%%%%%%%%%%%%%%%%%%%%%%%%%%%%%%%%%%%%%%%%%%%%%%%%
	%%%%%%%%%%%%%%%%%%%%%%%		REQUIREMENTS SPECIFICATION	     %%%%%%%%%%%%%%%%%%%%%%%%%%%%%%%%%%%%
	%%%%%%%%%%%%%%%%%%%%%%%%%%%%%%%%%%%%%%%%%%%%%%%%%%%%%%%%%%%%%%%%%%%%%%%%%%%%%%
\newpage
\section*{Specific Requirements}

	\subsection*{External Interface and Architectural Requirements}
		
		The NavUP system will be available for use on many platforms and make use of multiple means of newtork links and other extenal interfaces such as GPS and GIS services.
		Other requirements such as quality, integration and architectural constraints will also be covered in this section.
		
			\subsubsection*{Access Channel Requirements}
				From the typical user's perspective, us as humans, the NavUp system will largely be focused on mobile devices such as Android or iOS smartphones.
				These devices, will make use of location services and internet capabilities to send and recieve data to and from access channels and servers using mobile data,
				WiFi signals, crowdsourcing and geolocation systems such as GPS and GIS. The campus WiFi network will be the main form of system access as the signal
				strength covers most of the large area the University spans, and will help location accuracy indoors where the mobile devices may struggle to find decent mobile
				and location based signals.
				
			\subsubsection*{Quality Requirements}
				A system of the complexity and scale of NavUP must meet specific quality requirements such as:
				
				\begin{enumerate}
				\renewcommand{\labelenumi}{{\textbf{\arabic{enumi}.}}}
				\item Performance - the system must handle a possbility of over 50000 users simultaneously and still perform important functionality without failure
				\item Reliability - the sytem should perform its functionality reliably, most importantly the navigational requirements.
				\item Usability - the system should be easy to use while sporting an aesthetic and ergonomic interface.
				\item Integrability - the sustem should allow for additional features and modules to be added easily.
				\end{enumerate}
				
			\subsubsection{Integration Requirements}
				The NavUP system will be implemented simultaneously by all the students enrolled in COS 301, it is therefore imperative that the system complies blaa %more fluffling needed
			\subsubsection{Architecture Constraints}
				The clients ave specified the use of technologies such as GIS, UI/UX, Mobile Development, Wifi Networking, Real-time Data Analysis, Data Streaming and
				Persistence....
		
	\subsection*{Functional Requirements}
		bla
		
	\subsection*{Performance Requirements}
	\subsection*{Design Constraints}
	\subsection*{Software System Attributes}
    \subsection*{Other Requirements}
\end{document}


