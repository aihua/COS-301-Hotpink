%
%Latex Document made by Team HotPink for SRS, Round 1, Assignment 1, COS 301 2017
%

\documentclass[11pt,a4paper]{article}
\usepackage{geometry}
 \geometry{
 a4paper,
 total={170mm,257mm},
 left=25mm,
right=25mm,
 top=25mm,
 }

\title{Software Requirements Specification \\ NavUP \\ COS 301 \\ Team: Hotpink}
\date{Friday, February 24, 2017}
\author{Gregory Austin 14039712 \\ Stephanie Groutsch 14293324 \\ Timothy Kirker 11152402 \\ Drew Langley 11039753 \\ Xoliswa Ntshingila 13410378
\\ Melvin Zitha 12138747}


\usepackage[latin1]{inputenc}
\usepackage{amsmath}
\usepackage{amsfonts}
\usepackage{amssymb}

%\usepackage{fullpage}

\begin{document}
\maketitle
\newpage
\tableofcontents
%insert table of contents and figues etc

	%%%%%%%%%%%%%%%%%%%%%%%%%%%%%%%%%%%%%%%%%%%%%%%%%%%%%%%%%%%%%%%%%%%%%%%%%%%%%%
	%%%%%%%%%%%%%%%%%%%%%%%		INTRODUCTION	     %%%%%%%%%%%%%%%%%%%%%%%%%%%%%%%%%%%%%%%%%%
	%%%%%%%%%%%%%%%%%%%%%%%%%%%%%%%%%%%%%%%%%%%%%%%%%%%%%%%%%%%%%%%%%%%%%%%%%%%%%%
\newpage
\section{Introduction}
	\subsection{Purpose}
	The purpose of this document is to provide a detailed description of the requirements for the NavUP application, a navigation application which is intended to assist students , staff and the  public with navigating the University of Pretoria. Additionally it will outline and explain the complete plan for building the NavUP system.The explanations include the constraints for the system and how the system will interact with external applications. Ultimately this document is to be used for client approval such that the system may be implemented.

	\subsection{Scope}
<<<<<<< HEAD
The NavUP mobile application makes use of the Wi-Fi on campus in order to assist the user in finding the route to their destination from their current location or from a location which is on the University of Pretoria facilities this includes but is not limited to lecture halls, food courts, ablution facilities or possibly even sport or faculty houses. 
\\

The NavUP mobile application will use the users GPS location, on their smart devices, in conjunction with all the Wi-Fi routers on campus in order to determine the best route from their starting location. As a result the users require Wi-Fi enabled devices and login credentials to the Universities Wi-Fi network.The system will not be able to navigate a user from a location which is not within the range of the campus Wi-Fi.
\\

 Further more the  NavUP system also include best route capabilities, optimal routes for users with disabilities, status of the number or people currently on campus based on the number of devices currently connected to the Wi-Fi network,points of interest, activities and events that are happening on campus. It will also track the activity of the user e.g The number of steps taken and reward the user based on the goals  they have achieved.

	\subsection{Definitions, Acronyms and Abbreviations}
	

	\subsection{References}


	\subsection{Overview}
	These are the chapters that follow. The Overall description chapter which describes the overall system , the modules and the interfaces that exist within the system , the system functions, the charecteristics of the users of the system and the system constraints. The Specific Requirements chapter is intended for developers and contains the external interface requirements , functional requirements , performance capabilities of the system as well as the design constraints and quality related requirements.
=======
	The NavUP mobile application makes use of the Wi-Fi on campus in order to assist the user in finding the optimal route to their destination, whether it be to lecture halls, food courts, ablution facilities or possibly even sport or faculty houses.
	\\
\par
	The NavUP system will also include best route capabilities as well as allowing users with disabilities to navigate routes that are accessible to them. The NavUP system will also include points of interest as well as any activities that may be happening on campus.
\\
\par
	The NavUP mobile application will use the users GPS location, on their phones, in conjunction with all the Wi-Fi routers on campus in order to determine the best route from their starting location. All of the users phones should be wifi enabled and have access to Universitys Wi-Fi login information.
	
		
>>>>>>> origin/master

	%%%%%%%%%%%%%%%%%%%%%%%%%%%%%%%%%%%%%%%%%%%%%%%%%%%%%%%%%%%%%%%%%%%%%%%%%%%%%%
	%%%%%%%%%%%%%%%%%%%%%%%		OVERALL DESCRIPTION	     %%%%%%%%%%%%%%%%%%%%%%%%%%%%%%%%%%%%%%%
	%%%%%%%%%%%%%%%%%%%%%%%%%%%%%%%%%%%%%%%%%%%%%%%%%%%%%%%%%%%%%%%%%%%%%%%%%%%%%%
\newpage
\section{Overall Description}
	\subsection{Product Perspective}
	Similar to existing navigation applications such as: Google Maps, Apple Maps and Waze, NavUP is a standalone mobile application that acts as a campus navigation system which will be open for all students to use. It encompasses all technologies, which will be required for the necessary functionality as described in the product functions section. 
\\
\par
The application will be implemented using relevant hardware interfaces such as, but not necessarily, Wi-Fi, Cellular Data and GPS. Implementation of the hardware interfaces will require already existing software interfaces/OS interfaces to access data from the hardware interfaces relating to the main functionality of the NavUP application. 
\\
\par
The application will have to use communication interfaces to determine traffic and the best routes to follow, these communication interfaces will work through the required hardware interfaces. The main operations of the application will work through the user interface, which will work through all the underlying required technologies. These operations include: navigating to a class, showing where to go, describing where and what it is you’re going to. The DBMS will have to be manually maintained with class and event data or somehow interfaced with an existing DBMS or DSS that the university already has. 

	\subsection{Product Functions}
	The NavUP mobile application will support a variety of functions namely; navigation, providing information, allowing for users to decide on different routes, achieving goals and personalistaion capabilities.
	\\

	The navigation functionality will consist of locating the user and then navigating them to their required location. They will be provided with searching capabilities in order to obtain their required location. Users will then be able to save their destination for future use.
	\\

	Information will be provided to users when they are near points of interest. This information will consist of historical and general information regarding venues or landmarks. Information regarding cultural and sporting events that relate to the user will also be shown when appropriate.
	\\

	Different routes will be provided based on the users needs. Such routes will include the fastest path from one location to the next. This path will be determined based on the amount of pedestrian traffic. Another route will be provided for special needs users in order to allow them to reach their destination using the easiest path for their disability.
	\\

	Weekly goals will be provided to motivate users to attend all their lectures during the week as well as to increase the amount of steps they take.

	\subsection{User Characteristics}
	There are three main categories of individuals that will make use of the software namely administrators, general users and information providers.
	\\


	The administrators will ensure that the software works as intended and will be in charge of maintaining and upgrading the application. These users will need to understand how the system works as well as have programming knowledge in order to work on the software.
	\\

	The general users of the system will consist of students, lecturers, guests and university employees. They will only make use of the mobile application. This means that they will not require any knowledge of how the software is implemented however, they will need to have knowledge regarding how to use a mobile phone. These users will mainly use the application to navigate them 		from one place in the university to the next.
	\\

	The information providers role will be to inform administrators regarding events taking place at the university as well as to provide information regarding different buildings and landmarks within the university. They will not necessarily make use of the mobile application and will therefore only require knowledge regarding the university itself.

	\subsection{Constraints}
	The following are restrictions related to the application:
		\begin{enumerate}
				\renewcommand{\labelenumi}{{\textbf{\arabic{enumi}.}}}
				\item Security  - The users personal information and current location should not be accessible to the public.
				\item Accuracy - The users location should be found whether the user is indoors or not. The location of the user should also be found in terms of what floor of a building they are on.
				\item Performance - The system should be able to handle a large amount of users making use of the software at the same time.
				\item Reliability - The application should still operate when one or more data access points are no longer available.
				\item Accessibility - The application should be easily accessible to everyone that requires it.
				\item Usability - The applications interface should be easy to use and understand.
				\item Size - The application should not require too much memory in order to operate.
				\end{enumerate}
	\subsection{Assumptions and Dependencies}
	Since the NavUP system is being created for mobile devices, it is assumed that all those who require the application, will have access to a mobile device. It is then assumed that these devices will have the required capabilities necessary to run the application. Such dependencies include built in WiFi, GPS and cellular connectivity as well as the required data/airtime.
\\
\par
Assumed users will be registered students of the University of Pretoria thereby having access to their modules and class data for navigating when required with minimal input from the user. This depends on access to students data for the DBMS.
\\
\par
In terms of the usability of the application, it is assumed that users will have knowledge regarding how to use mobile devices and applications on such devices.
\\
\par
The assumption relating to accuracy includes the belief that the application will be up to date with the latest information in terms of venues, landmarks and events. This assumes that there will always be someone maintaining and updating the applications data, or a DBMS that is interfaced with the university DBMS.  
\\
\par
An assumption in terms of the systems performance includes the idea that it will still be able to function when a large amount of users are making use of the application at the same time. This depends on server capacity as well as solid code structures that are efficient as well as safe. 

	
	%%%%%%%%%%%%%%%%%%%%%%%%%%%%%%%%%%%%%%%%%%%%%%%%%%%%%%%%%%%%%%%%%%%%%%%%%%%%%%
	%%%%%%%%%%%%%%%%%%%%%%%		REQUIREMENTS SPECIFICATION	     %%%%%%%%%%%%%%%%%%%%%%%%%%%%%%%%%%%%
	%%%%%%%%%%%%%%%%%%%%%%%%%%%%%%%%%%%%%%%%%%%%%%%%%%%%%%%%%%%%%%%%%%%%%%%%%%%%%%
\newpage
\section{Requirements Specification}

	\subsection{External Interface and Architectural Requirements}
		
		The NavUP system will be available for use on many platforms and make use of multiple means of newtork links and other extenal interfaces such as GPS and GIS services.
		Other requirements such as quality, integration and architectural constraints will also be covered in this section.
		
			\subsubsection{Access Channel Requirements}
				From the typical user's perspective, approximately 60000 students and staff, the NavUP system will largely be focused on mobile devices such as Android or iOS smartphones.
				These devices, will make use of location services and internet capabilities to send and recieve data to and from access channels and servers using mobile data,
				Wi-Fi signals, crowdsourcing and geolocation systems such as GPS and GIS. The campus Wi-Fi network will be the main form of system access as the signal
				strength covers most of the large area the University spans, and will help location accuracy indoors where the mobile devices may struggle to find decent mobile
				and location based signals.
				
			\subsubsection{Quality Requirements}
				A system of the complexity and scale of NavUP must meet specific quality requirements such as:
				
				\begin{enumerate}
				\renewcommand{\labelenumi}{{\textbf{\arabic{enumi}.}}}
				\item Performance - the system must handle a possbility of over 50000 users simultaneously and still perform important functionality without failure
				\item Reliability - the sytem should perform its functionality reliably, most importantly the navigational requirements.
				\item Usability - the system should be easy to use while sporting an aesthetic and ergonomic interface.
				\item Integrability - the system should allow for additional features and modules to be added easily.
				\end{enumerate}
				
			\subsubsection{Integration Requirements}
				The NavUP system will be implemented simultaneously by all the students enrolled in COS 301, it is therefore imperative that the system complies to a low coupling and high cohesion standard.
			\subsubsection{Architecture Constraints}
				The clients ave specified the use of technologies such as GIS, UI/UX, Mobile Development (Android/iOS),  Real-time Data Analysis, Data Streaming and
				Persistence, and a large focus on the use of Wi-Fi Networking to boost the system's accuracy. A module that supplies targeted delivery of information was also requested hence an intelligent program is also required.
				
			\subsubsection{User Interface}
				The user interface is likely to be used by students and staff alike, and many visitors to the campus, the interface therefore should be friendly and easy to use for all ages. The interface should include functionality for users to login and create personal profiles
				so that they can use the rewards system and access other functionality. There should also be different interface for administrative users who must be able to create point of interest and update activities and update and capture location data.
	\newpage	
	\subsection{Functional Requirements}
		Listed below are non-trivial use cases in order ranging from critical through nice-to-have. The logical structure of the data is contained at the end of the section.
		
			\subsubsection{Use Case Name: Navigate user to required location}
				\begin{enumerate}
				\renewcommand{\labelenumi}{{\textbf{\arabic{enumi}.}}}
				\item Xref????  %needed or even necessary?
				\item Trigger: user inputs a desired destination
				\item Precondition: user is logged in and enters the navigation module.
				\item Basic Path: user inputs a destination which is sent to the navigation module after which an optimal route to the destination is calculated by the navigation and traffic modules and sent back to the client device which will display the infrormation.
				\item Postcondition: client device recieves navigational information to destination.
				\end{enumerate}
				
			\subsubsection{Use Case Name: allow users the ability to save a location}
				\begin{enumerate}
				\renewcommand{\labelenumi}{{\textbf{\arabic{enumi}.}}}
				\item Xref????  %needed or even necessary?
				\item Trigger: user taps/presses  the button for save/share destination.
				\item Precondition: user is logged in and wishes to save/share a location.
				\item Basic Path: after pressing the save/share location button, the client device will send data regarding the users current position. This information is persisted to the users database and stored for future use by the user.
				\item Postcondition: client device displays storage of location by showing an icon over the users current position.
				\end{enumerate}
				
			\subsubsection{Use Case Name: Display metadata with information regarding points of interest}
				\begin{enumerate}
				\renewcommand{\labelenumi}{{\textbf{\arabic{enumi}.}}}
				\item Xref????  %needed or even necessary?
				\item Trigger: user navigates passed or near a point of interest.
				\item Precondition: user is logged in and is currently en route to a destination.
				\item Basic Path: user navigated passed a point of interest, the client application sends current data to the points of interest module which searches for any nearby points. After which any data gathered is persisted back to the client device which displays it in the form of a popup.
				\item Postcondition: client device displays popup of information regarding the point/points of interest.
				\end{enumerate}
				
			\subsubsection{Use Case Name: Give user an ETA to destination}
				\begin{enumerate}
				\renewcommand{\labelenumi}{{\textbf{\arabic{enumi}.}}}
				\item Xref????  %needed or even necessary?
				\item Trigger: user enters a destination and is en route.
				\item Precondition: user is logged in and is currently travelling to a desired destination.
				\item Basic Path: information regarding the users current location and destination are sent to the navigation and traffic modules which calculate the users estimated time of arrival at the destination which is sent back to the client device.
				\item Postcondition: client device recieves and displays estimated time of arrival at the destination.
				\end{enumerate}
				
			\subsubsection{Use Case Name: providing information about venues and activities }
				\begin{enumerate}
				\renewcommand{\labelenumi}{{\textbf{\arabic{enumi}.}}}
				\item Xref????  %needed or even necessary?
				\item Trigger: user requests information about venue, cultural or sporting activities.
				\item Precondition: user is logged in and requests to view activities.
				\item Basic Path: the request is sent to the activities database which searches and returns all feasible and viable activities matching the users search parameters, which is then sent back to the clients device.
				\item Postcondition: client device recieves and displays information regarding the requested activities.
				\end{enumerate}
				
			\subsubsection{Use Case Name: provide special routes for users with disabilities}
				\begin{enumerate}
				\renewcommand{\labelenumi}{{\textbf{\arabic{enumi}.}}}
				\item Xref????  %needed or even necessary?
				\item Trigger: user inputs a desired destination.
				\item Precondition: user is logged in and is disabled, this is set by additional boolean options in the user database.
				\item Basic Path: the destination and current location of the user is sent to the navigational and traffic modules which calculate an optimal route to the destination, with the addition of special features such as wheelchair ramps.
				\item Postcondition: client device recieves and displays information regarding the optimal navigational route.
				\end{enumerate}
				
			\subsubsection{Use Case Name:Administrative, create points of interest}
				\begin{enumerate}
				\renewcommand{\labelenumi}{{\textbf{\arabic{enumi}.}}}
				\item Xref????  %needed or even necessary?
				\item Trigger: user desires to create a point of interest.
				\item Precondition: user is logged in as an administrator, this is set by additional boolean options in the user database.
				\item Basic Path: the create point of interest button is pressed, which displays a popup notification asking for information about the point of interest such as the name and location etc. This information will be persisted to the database and displayed on all devices.
				\item Postcondition: client device recieves notification of successful creation of a point of interest, which is then displayed in naviigational information.
				\end{enumerate}
				
			\subsubsection{Use Case Name:collecting and viewing rewards}
				\begin{enumerate}
				\renewcommand{\labelenumi}{{\textbf{\arabic{enumi}.}}}
				\item Xref????  %needed or even necessary?
				\item Trigger: user completes an activity or reward worthy action such as x amount of steps/meters weekly.
				\item Precondition: user is logged in and completes an activity.
				\item Basic Path: a module for the rewards system triggers events during constant monitoring and persistence of data between client and server so as to notify the user when an activity has been completed, the system sends a notification to the client device upon activity completion.
				\item Postcondition: client recieves notification about completion of an activity and is directed to the third party rewards system.
				\end{enumerate}

\end{document}
